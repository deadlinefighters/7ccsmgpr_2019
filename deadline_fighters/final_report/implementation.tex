\section{Implementation}

The initial team of 6 was divided into sub-teams of 2, each handling desktop client, mobile client and the server component. At this point, server component was AWS S3 and the client applications made direct calls. After team restructuring, development of local server began while the client applications continued to build the system to talk to AWS S3. Desktop, mobile and server code were pushed into corresponding feature branch and then merged into the master branch at regular intervals. This allowed for better testing and narrowing down of issues.\newline

Once the basic decisions about the server were made and functionalities implemented, client applications were migrated to interact through the local server. This delayed progress which we could have used into feature building if we had made better decisions on the server.\newline

Development process was agile with weekly scrum meetings on Thurdays. Emphasis was laid on writing maintainable features and code. Efforts were directed into code reviews, logging for client and server and documenting commonly faced issues. Bugs were tracked under \emph{issues} on our Git repository.\newline

The following are the versions of softwares, packages and IDEs used:\newline

\textbf{Desktop Application Development}
\begin{itemize}
	\item Framework: Electron 4.0.1
	\item Dependencies: Chokidar (for file/folder watching), dns, ip, httpreq (for http request)\newline
\end{itemize}

\textbf{Mobile Application Development}
\begin{itemize}
	\item IDE: Android Studio 3.3
	\item Required API: equal to or greater than 25
	\item Build system: Gradle\newline
\end{itemize}

\textbf{Server Development}
\begin{itemize}
	\item Language: Python 2.7
	\item Database: Sqlite\newline
\end{itemize}

\textbf{Cloud endpoint}
\begin{itemize}
	\item Service: Amazon Web Services Simple Storage Service (AWS S3)\newline
\end{itemize}